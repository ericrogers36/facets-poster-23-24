% Unofficial University of Edinburgh
% https://github.com/andiac/gemini-cam
% a fork of https://github.com/anishathalye/gemini
% also refer to https://github.com/k4rtik/uchicago-poster
% Adapted from the University of Cambridge Version

\documentclass[final]{beamer} 

% ====================
% Packages
% ====================

\usepackage[T1]{fontenc}
\usepackage{lmodern}
%\usepackage[size=custom,width=120,height=72,scale=1.0]{beamerposter}
\usepackage[size=custom,width=120,height=72,scale=1.25]{beamerposter}
%\usepackage[orientation=landscape,size=a1,scale=1.4]{beamerposter}
%\usepackage[size=custom,width=108,height=63,scale=1.1]{beamerposter}
\usetheme{gemini}
\usecolortheme{customuoe}
\usepackage{graphicx}
\usepackage{booktabs}
\usepackage{tikz}
\usepackage{pgfplots}
\pgfplotsset{compat=1.14}
\usepackage{anyfontsize}

% ====================
% Lengths
% ====================

% If you have N columns, choose \sepwidth and \colwidth such that
% (N+1)*\sepwidth + N*\colwidth = \paperwidth
\newlength{\sepwidth}
\newlength{\colwidth}
\setlength{\sepwidth}{0.025\paperwidth}
\setlength{\colwidth}{0.3\paperwidth}

\newcommand{\separatorcolumn}{\begin{column}{\sepwidth}\end{column}}

% ====================
% Title
% ====================

\title{Projective Geometry and Dobble}

\author{Ben Jecock, Corinne Baillie, Eric Rogers, Samuel Buxton, Teagan Addy}

% ====================
% Footer (optional)
% ====================

%\footercontent{
%  \href{https://www.example.com}{https://www.example.com} \hfill
%  ABC Conference 2025, New York --- XYZ-1234 \hfill
%  \href{mailto:alyssa.p.hacker@example.com}{alyssa.p.hacker@example.com}}
% (can be left out to remove footer)

% ====================
% Logo (optional)
% ====================

% use this to include logos on the left and/or right side of the header:
% \logoright{\includegraphics[height=7cm]{logo1.pdf}}
%\logoleft{\includegraphics[height=7cm]{logos/uoelogo.png}}

% ====================
% Body
% ====================

\begin{document}

% Refer to https://github.com/k4rtik/uchicago-poster
% logo: https://www.cam.ac.uk/brand-resources/about-the-logo/logo-downloads
\addtobeamertemplate{headline}{}
{
    \begin{tikzpicture}[remember picture,overlay]
      \node [anchor=north west, inner sep=3cm] at ([xshift=-1.0cm,yshift=1.0cm]current page.north west)
      {\includegraphics[height=4.5cm]{logos/uoelogo.png}}; 
    \end{tikzpicture}
}

\begin{frame}[t]
\begin{columns}[t]
\separatorcolumn

\begin{column}{\colwidth}

  \begin{block}{Introduction}
    Dobble is a game created by Jacques Cottereau - a French mathematics enthusiast. Cottereau’s idea for the game came from a fascination for combinatorics, pairs and ordering and is said to be inspired by Kirkman’s schoolgirl problem. There are $55$ cards in a standard game of Dobble, each containing $8$ distinct symbols from a pool of $57$ unique symbols. The game has taken many different forms - some versions aim to get rid of your cards first, some to collect as many as possible, or even to sabotage your opponents. However, the key principle of the game is that every unique pair of Dobble cards contain one symbol in common. Although this fact seems like a coincidence, it relies on aspects of geometry, combinatorics, the projective - and finite projective - plane. 
  \end{block}

  \begin{block}{The Dobble Problem and Combinatorics}

    Let us first imagine that we have a big set of unique items. We wish to have subsets of this big set which contain a fixed, equal number of items within them.
    
    The rule is that these subsets have to have an overlap of \emph{one item} - there cannot be an overlap of two unique items between any two subsets. In effect, for any pair of items in any subset, \emph{this pair cannot reappear in any other subset}.

    This is known as a Steiner system notated as $S(2,k,n)$, where $k$ is the number of blocks and $n$ is the number of total items. (It can also be referred to as a symmetric $2$-design, a $2$-$(n,k,1)$ design).

This problem is very much akin to Dobble - in essence, the items are points and the subsets are cards. For the standard verison of Dobble, we essentially just have the Steiner system $S(2,8,57)$.
These parameters are because:
\begin{itemize}
    \item We have 57 cards total in Dobble
    \item There are cards (blocks) with $8$ symbols with the constraint that
    \item Any $2$ pictures are contained in exactly one card (block).
\end{itemize}

However, there exists a link between Steiner systems and projective planes. If we have a finite projective plane $P$ such that any line has exactly $q+1$ points, we can conclude that there are $q^2+q+1$ total points in $P$, and that $P$ is a $S(2, q+ 1, q^2 +q+ 1)$
Steiner system (or a $2$-$(q^2 +q+ 1, q + 1, 1)$ design)\cite{storme2006}.
  \end{block}

  \begin{alertblock}{A highlighted block}

    This block catches your eye, so \textbf{important stuff} should probably go
    here.

    Curabitur eu libero vehicula, cursus est fringilla, luctus est. Morbi
    consectetur mauris quam, at finibus elit auctor ac. Aliquam erat volutpat.
    Aenean at nisl ut ex ullamcorper eleifend et eu augue. Aenean quis velit
    tristique odio convallis ultrices a ac odio.

  \end{alertblock}

\end{column}

\separatorcolumn

\begin{column}{\colwidth}

  \begin{block}{A block containing an enumerated list}

    Vivamus congue volutpat elit non semper. Praesent molestie nec erat ac
    interdum. In quis suscipit erat. \textbf{Phasellus mauris felis, molestie
    ac pharetra quis}, tempus nec ante. Donec finibus ante vel purus mollis
    fermentum. Sed felis mi, pharetra eget nibh a, feugiat eleifend dolor. Nam
    mollis condimentum purus quis sodales. Nullam eu felis eu nulla eleifend
    bibendum nec eu lorem. Vivamus felis velit, volutpat ut facilisis ac,
    commodo in metus.

    \begin{enumerate}
      \item \textbf{Morbi mauris purus}, egestas at vehicula et, convallis
        accumsan orci. Orci varius natoque penatibus et magnis dis parturient
        montes, nascetur ridiculus mus.
      \item \textbf{Cras vehicula blandit urna ut maximus}. Aliquam blandit nec
        massa ac sollicitudin. Curabitur cursus, metus nec imperdiet bibendum,
        velit lectus faucibus dolor, quis gravida metus mauris gravida turpis.
      \item \textbf{Vestibulum et massa diam}. Phasellus fermentum augue non
        nulla accumsan, non rhoncus lectus condimentum.
    \end{enumerate}

  \end{block}

    \begin{block}{A block title}

    Some block contents, followed by a diagram, followed by a dummy paragraph.

    \begin{figure}
      \centering
      \begin{tikzpicture}[scale=3]
  \draw \foreach \a in {30,150,270}{(\a:1)  -- (180+\a:2)}
        (90:2)  -- (210:2) -- (330:2) -- cycle
        (0:0) circle (1);
  \fill \foreach \p in {(0:0),(30:1),(90:2),(150:1),(210:2),(270:1),(330:2)}
        {\p circle(3pt)};
\end{tikzpicture}
      \caption{The Fano Plane}
    \end{figure}

    Lorem ipsum dolor sit amet, consectetur adipiscing elit. Morbi ultricies
    eget libero ac ullamcorper. Integer et euismod ante. Aenean vestibulum
    lobortis augue, ut lobortis turpis rhoncus sed. Proin feugiat nibh a
    lacinia dignissim. Proin scelerisque, risus eget tempor fermentum, ex
    turpis condimentum urna, quis malesuada sapien arcu eu purus.

  \end{block}

  \begin{block}{Nam cursus consequat egestas}

    Nulla eget sem quam. Ut aliquam volutpat nisi vestibulum convallis. Nunc a
    lectus et eros facilisis hendrerit eu non urna. Interdum et malesuada fames
    ac ante \textit{ipsum primis} in faucibus. Etiam sit amet velit eget sem
    euismod tristique. Praesent enim erat, porta vel mattis sed, pharetra sed
    ipsum. Morbi commodo condimentum massa, \textit{tempus venenatis} massa
    hendrerit quis. Maecenas sed porta est. Praesent mollis interdum lectus,
    sit amet sollicitudin risus tincidunt non.

  \end{block}

\end{column}

\separatorcolumn

\begin{column}{\colwidth}

  \begin{exampleblock}{A highlighted block containing some math}

    A different kind of highlighted block.

    $$
    \int_{-\infty}^{\infty} e^{-x^2}\,dx = \sqrt{\pi}
    $$

    Interdum et malesuada fames $\{1, 4, 9, \ldots\}$ ac ante ipsum primis in
    faucibus. Cras eleifend dolor eu nulla suscipit suscipit. Sed lobortis non
    felis id vulputate.

    \heading{A heading inside a block}

    Praesent consectetur mi $x^2 + y^2$ metus, nec vestibulum justo viverra
    nec. Proin eget nulla pretium, egestas magna aliquam, mollis neque. Vivamus
    dictum $\mathbf{u}^\intercal\mathbf{v}$ sagittis odio, vel porta erat
    congue sed. Maecenas ut dolor quis arcu auctor porttitor.

    \heading{Another heading inside a block}

    Sed augue erat, scelerisque a purus ultricies, placerat porttitor neque.
    Donec $P(y \mid x)$ fermentum consectetur $\nabla_x P(y \mid x)$ sapien
    sagittis egestas. Duis eget leo euismod nunc viverra imperdiet nec id
    justo.

  \end{exampleblock}

  \begin{block}{Nullam vel erat at velit convallis laoreet}

    Class aptent taciti sociosqu ad litora torquent per conubia nostra, per
    inceptos himenaeos. Phasellus libero enim, gravida sed erat sit amet,
    scelerisque congue diam. Fusce dapibus dui ut augue pulvinar iaculis.

    \begin{table}
      \centering
      \begin{tabular}{l r r c}
        \toprule
        \textbf{First column} & \textbf{Second column} & \textbf{Third column} & \textbf{Fourth} \\
        \midrule
        Foo & 13.37 & 384,394 & $\alpha$ \\
        Bar & 2.17 & 1,392 & $\beta$ \\
        Baz & 3.14 & 83,742 & $\delta$ \\
        Qux & 7.59 & 974 & $\gamma$ \\
        \bottomrule
      \end{tabular}
      \caption{A table caption.}
    \end{table}

  \end{block}

  \begin{block}{References}
   \nocite{*}
   \footnotesize{\bibliographystyle{plain}\bibliography{poster}}

  \end{block}

\end{column}

\separatorcolumn
\end{columns}
\end{frame}

\end{document}
