% Unofficial University of Edinburgh
% https://github.com/andiac/gemini-cam
% a fork of https://github.com/anishathalye/gemini
% also refer to https://github.com/k4rtik/uchicago-poster
% Adapted from the University of Cambridge Version

\documentclass[final]{beamer} 

% ====================
% Packages
% ====================

\usepackage[T1]{fontenc}
\usepackage{lmodern}
%\usepackage[size=custom,width=120,height=72,scale=1.0]{beamerposter}
\usepackage[size=custom,width=120,height=72,scale=1.25]{beamerposter}
%\usepackage[orientation=landscape,size=a1,scale=1.4]{beamerposter}
%\usepackage[size=custom,width=108,height=63,scale=1.1]{beamerposter}
\usetheme{gemini}
\usecolortheme{customuoe}
\usepackage{graphicx}
\usepackage{booktabs}
\usepackage{tikz}
\usepackage{pgfplots}
\pgfplotsset{compat=1.14}
\usepackage{anyfontsize}

% ====================
% Lengths
% ====================

% If you have N columns, choose \sepwidth and \colwidth such that
% (N+1)*\sepwidth + N*\colwidth = \paperwidth
\newlength{\sepwidth}
\newlength{\colwidth}
\setlength{\sepwidth}{0.025\paperwidth}
\setlength{\colwidth}{0.3\paperwidth}

\newcommand{\separatorcolumn}{\begin{column}{\sepwidth}\end{column}}

% ====================
% Title
% ====================

\title{Projective Geometry and Dobble}

\author{Ben Jecock, Corinne Baillie, Eric Rogers, Samuel Buxton, Teagan Addy}

% ====================
% Footer (optional)
% ====================

%\footercontent{
%  \href{https://www.example.com}{https://www.example.com} \hfill
%  ABC Conference 2025, New York --- XYZ-1234 \hfill
%  \href{mailto:alyssa.p.hacker@example.com}{alyssa.p.hacker@example.com}}
% (can be left out to remove footer)

% ====================
% Logo (optional)
% ====================

% use this to include logos on the left and/or right side of the header:
% \logoright{\includegraphics[height=7cm]{logo1.pdf}}
%\logoleft{\includegraphics[height=7cm]{logos/uoelogo.png}}

% ====================
% Body
% ====================

\begin{document}

% Refer to https://github.com/k4rtik/uchicago-poster
% logo: https://www.cam.ac.uk/brand-resources/about-the-logo/logo-downloads
\addtobeamertemplate{headline}{}
{
    \begin{tikzpicture}[remember picture,overlay]
      \node [anchor=north west, inner sep=3cm] at ([xshift=-1.0cm,yshift=1.0cm]current page.north west)
      {\includegraphics[height=4.5cm]{logos/uoelogo.png}}; 
    \end{tikzpicture}
}

\begin{frame}[t]
\begin{columns}[t]
\separatorcolumn

\begin{column}{\colwidth}

  \begin{block}{Introduction}
    Dobble is a game created by Jacques Cottereau - a French mathematics enthusiast. Cottereau’s idea for the game came from a fascination for combinatorics, pairs and ordering and is said to be inspired by Kirkman’s schoolgirl problem. There are $55$ cards in a standard game of Dobble, each containing $8$ distinct symbols from a pool of $57$ unique symbols. The game has taken many different forms - some versions aim to get rid of your cards first, some to collect as many as possible, or even to sabotage your opponents. However, the key principle of the game is that every unique pair of Dobble cards contain one symbol in common. Although this fact seems like a coincidence, it relies on aspects of geometry, combinatorics, the projective - and finite projective - plane. 
  \end{block}

  \begin{block}{The Dobble Problem and Combinatorics}

    Let us first imagine that we have a big set of unique items. We wish to have subsets of this big set which contain a fixed, equal number of items within them.
    
    The rule is that these subsets have to have an overlap of \emph{one item} - there cannot be an overlap of two unique items between any two subsets. In effect, for any pair of items in any subset, \emph{this pair cannot reappear in any other subset}.

    This is known as a Steiner system notated as $S(2,k,n)$, where $k$ is the number of blocks and $n$ is the number of total items. (It can also be referred to as a symmetric $2$-design, a $2$-$(n,k,1)$ design).

This problem is very much akin to Dobble - in essence, the items are points and the subsets are cards. For the standard verison of Dobble, we essentially just have the Steiner system $S(2,8,57)$.
These parameters are because:
\begin{itemize}
    \item We have 57 cards total in Dobble
    \item There are cards (blocks) with $8$ symbols with the constraint that
    \item Any pair of $2$ pictures are contained in \emph{exactly} one card (block).
\end{itemize}

However, there exists a link between Steiner systems and projective planes. If we have a finite projective plane $P$ such that any line has exactly $q+1$ points, we can conclude that there are $q^2+q+1$ total points in $P$, and that $P$ is a $S(2, q+ 1, q^2 +q+ 1)$
Steiner system (or a $2$-$(q^2 +q+ 1, q + 1, 1)$ design) \cite{storme2006}.
  \end{block}

  \begin{alertblock}{A highlighted block}

    This block catches your eye, so \textbf{important stuff} should probably go
    here.

    Curabitur eu libero vehicula, cursus est fringilla, luctus est. Morbi
    consectetur mauris quam, at finibus elit auctor ac. Aliquam erat volutpat.
    Aenean at nisl ut ex ullamcorper eleifend et eu augue. Aenean quis velit
    tristique odio convallis ultrices a ac odio.

  \end{alertblock}

\end{column}

\separatorcolumn

\begin{column}{\colwidth}

  \begin{block}{Steiner systems and Projective Planes}

    We can delve deeper into their relation to Dobble.
Consider a game of Dobble with only $7$ cards. If each point represents a card, lines can be drawn such that if we were to chose any two of the seven points there is a line that connects them. We can say that these lines represent matching symbols on each card. So if we draw this configuration for the 7 card example we get:
    \begin{figure}
      \centering
      \begin{tikzpicture}[scale=3]
  \draw \foreach \a in {30,150,270}{(\a:1)  -- (180+\a:2)}
        (90:2)  -- (210:2) -- (330:2) -- cycle
        (0:0) circle (1);
  \fill \foreach \p in {(0:0),(30:1),(90:2),(150:1),(210:2),(270:1),(330:2)}
        {\p circle(3pt)};
\end{tikzpicture}
      \caption{7 Card Dobble - The Fano Plane}
    \end{figure}
It follows that each card contains $3$ symbols, no two cards are the same and any two cards will have one and only one matching symbol with $7$ different symbols in total.
We can see that this too is a Steiner system where, $q = p^n = 2^1$ , $q + 1 = 3 =$ no. of symbols on each card, $q^2 + q + 1 = 7 =$ no. of cards and total no. of symbols

Now we can apply this to our full sized version of Dobble.
We can start by constructing a $7\times7$ grid of points and lines such that if we choose any two of the $49$ points there is a line that connects them. This can be achieved by '' Wrapping around `` the grid until it goes through $7$ points.
We are then left with a grid of $56$ lines and we can see that there are actually $8$ different systems of $7$ parallel lines. Each collection of parallel lines converges at infinity at some point,
hence there are $8$ more points at infinity that can be connected with a final line. Therefore, all lines go through $8$ points.
Once again this visualises the fact that it is a Steiner system. Where $q = p^n = 7^1$ , $q + 1 = 8 =$ no. of symbols on each card , $q^2 + q + 1 = 57 =$ no. of cards and total no. of symbols

It is theorised that to simplify the printing process there are only $55$ cards in Dobble. Cards can still be arranged in the same way and essentially 'skip' these points and it will still work. 

  \end{block}



\end{column}

\separatorcolumn

\begin{column}{\colwidth}

  \begin{exampleblock}{A highlighted block containing some math}

    A different kind of highlighted block.

    $$
    \int_{-\infty}^{\infty} e^{-x^2}\,dx = \sqrt{\pi}
    $$

    Interdum et malesuada fames $\{1, 4, 9, \ldots\}$ ac ante ipsum primis in
    faucibus. Cras eleifend dolor eu nulla suscipit suscipit. Sed lobortis non
    felis id vulputate.

    \heading{A heading inside a block}

    Praesent consectetur mi $x^2 + y^2$ metus, nec vestibulum justo viverra
    nec. Proin eget nulla pretium, egestas magna aliquam, mollis neque. Vivamus
    dictum $\mathbf{u}^\intercal\mathbf{v}$ sagittis odio, vel porta erat
    congue sed. Maecenas ut dolor quis arcu auctor porttitor.

    \heading{Another heading inside a block}

    Sed augue erat, scelerisque a purus ultricies, placerat porttitor neque.
    Donec $P(y \mid x)$ fermentum consectetur $\nabla_x P(y \mid x)$ sapien
    sagittis egestas. Duis eget leo euismod nunc viverra imperdiet nec id
    justo.

  \end{exampleblock}

  \begin{block}{A block title}

    Some block contents, followed by a diagram, followed by a dummy paragraph.

    Lorem ipsum dolor sit amet, consectetur adipiscing elit. Morbi ultricies
    eget libero ac ullamcorper. Integer et euismod ante. Aenean vestibulum
    lobortis augue, ut lobortis turpis rhoncus sed. 
  \end{block}

  \begin{block}{References}
   \nocite{*}
   \footnotesize{\bibliographystyle{plain}\bibliography{poster}}

  \end{block}

\end{column}

\separatorcolumn
\end{columns}
\end{frame}

\end{document}
