\documentclass[25pt,a1paper,landscape,
               margin=0.5cm,colspace=1.25cm,
               subcolspace=-0.5cm,blockverticalspace=1.25cm]{tikzposter}
\usecolorstyle{Default}
\colorlet{backgroundcolor}{white}

\usepackage[utf8]{inputenc}
\usepackage{amsmath}
\usepackage{amssymb}
\usepackage{listings}
\usepackage{zref-savepos}

\newtheorem{definition}{Definition}
\newtheorem{theorem}{Theorem}

% Remove LaTeX / tikzposter information
\tikzposterlatexaffectionproofoff

% Title based on the example in section 4.1 of the tikzposter documentation
%   P. Richter, E. Botoeva, R. Barnard, and D. Surmann, "The tikzposter class", Jan. 17 2014
\settitle{ \centering \vbox{
    \vspace{1.25cm}
    \centering
    \color{titlefgcolor} {\fontsize{72pt}{72pt}\selectfont \bfseries \sc \@title\\}
    \vspace{1.25cm}
    % Reduce the font size here if necessary to fit all names
    {\fontsize{40pt}{40pt}\selectfont \@author}
    \vspace{1.25cm}
}}

% Defines a block which fills to the bottom of the poster
\newcounter{vfillblockcounter}
\newcommand{\vfillblock}[2]{
    \block{#1}{
        \zsaveposy{vfillblock\the\value{vfillblockcounter}}
        \parbox[t][\zposy{vfillblock\the\value{vfillblockcounter}}sp
                   - \the\TP@blockbodyinnersep - \the\TP@blockverticalspace
                   - \the\paperheight/2 + \the\textheight/2]{\linewidth}{#2}
    }
    \stepcounter{vfillblockcounter}
}

% Some custom colours
\definecolor{custom_grey_0}{rgb}{0.7, 0.7, 0.7}
\definecolor{custom_grey_1}{rgb}{0.2, 0.2, 0.2}

% Poster title
\title{Facets of Mathematics Poster Template}
% List of authors
\author{James R. Maddison}

\begin{document}

% The title
\maketitle[width=0.8\textwidth,roundedcorners=30pt,innersep=0cm,
           titletotopverticalspace=1.25cm,
           titletoblockverticalspace=1.25cm]

% The rest of the poster
\begin{columns}

% First column
\column{0.25}
% First block in the first column
\block{Poster creation with \LaTeX}{
This template can be used to create a poster using \LaTeX. It makes use of the \texttt{tikzposter} \LaTeX\ package to typeset the different poster elements.
}

% Second block in the first column
\block{Including mathematics}{
The full features of \LaTeX\ are available within the poster. For example we can include inline formul{\ae}, $e^{i \pi} + 1 = 0$, and can include numbered equations
\begin{equation}\label{eqn:forward_euler}
  \psi_{n + 1} = \psi_n + h F \left( \psi_n, t_n \right),
\end{equation}
and then refer to them via a label, as in equation \eqref{eqn:forward_euler}. We can also include unnumbered and multi-line equations.
}

% Final block in the first column
\vfillblock{Including code}{
It is possible to include code, but the code should appear in a separate file and be included with \texttt{lstinputlisting}. For example:
\bigskip
% Note the reduced font size and extra indentation here
\lstinputlisting[language=Python,basicstyle=\normalsize,xleftmargin=1cm]{factorial.py}
}

% Second column
\column{0.75}

% First block in the second column
\block{Creating columns}{
Columns are created using the \texttt{column} command inside the \texttt{columns} environment (i.e. between \texttt{{\textbackslash}begin\{columns\}} and \texttt{{\textbackslash}end\{columns\}}). Above this block (check the source code!) \texttt{{\textbackslash}column\{0.75\}} creates a column which spans three quarters of the poster. Poster contents is added to the columns by adding one or more blocks.

\bigskip
Subcolumns are created using the \texttt{{\textbackslash}subcolumn} command within the \texttt{subcolumns} environment (i.e. between \texttt{{\textbackslash}begin\{subcolumns\}} and \texttt{{\textbackslash}end\{subcolumns\}}). Below, \texttt{{\textbackslash}subcolumn\{0.33\}} creates a subcolumn which spans 33\% of the \emph{column} within which the subcolumns are defined.
}

% Divide the second column into subcolumns
\begin{subcolumns}

% First subcolumn
\subcolumn{0.33}

% First block in the first subcolumn
\block{Defining blocks}{
Within each column or subcolumn a block, containing text and other elements, can be defined using the \texttt{block} command. This can be used via \texttt{{\textbackslash}block\{[title]\}\{[contents]\}}.

\bigskip
The final block in a column or subcolumn can be expanded to fill to the bottom of the poster by using \texttt{{\textbackslash}vfillblock} in place of \texttt{{\textbackslash}block}.
}

% Final block in the second subcolumn
{  % Extra brackets here prevent these colours being used for all remaining blocks
\colorlet{blocktitlefgcolor}{black}
\colorlet{blocktitlebgcolor}{custom_grey_0}
\colorlet{blockbodyfgcolor}{white}
\colorlet{blockbodybgcolor}{custom_grey_1}
\vfillblock{Customizing blocks}{
It is possible to change the colours used in a block -- for example to highlight a particularly important piece of information or a particularly important result. This example sets all four colours, and also makes use of some custom colours set near the top of the file using \texttt{definecolor}.
}
}

% Second subcolumn
\subcolumn{0.67}

% First block in the second subcolumn
\block{Including figures}{
\begin{minipage}[m]{0.4\linewidth}
You can add figures to the poster, but this is a bit different from the documents we have used so far. To include a figure in the poster you should use the \texttt{tikzfigure} environment, with the caption in square brackets, and the label before the figure itself (see the source code for Fig. \ref{fig:chess}).

\bigskip
In this example we also use the \texttt{minipage} environment to place this text to the left of the image.
\end{minipage}
\begin{minipage}[m]{0.6\linewidth}
\begin{tikzfigure}[An $8$-bit greyscale image of part of a chess set.]
\label{fig:chess}
\includegraphics[width=0.8\linewidth]{chess_1024x768_gs.png}
\end{tikzfigure}
\end{minipage}
}

% Final block in the second subcolumn
\vfillblock{References}{
{
\fontsize{26}{26}\selectfont  % Change the font size for the references
\renewcommand{\section}[2]{}  % Prevent any new sections (i.e. 'References') being created here
% The bibliography
\begin{thebibliography}{9}
% A journal article
\bibitem{einstein1905}
A. Einstein,
\emph{Ist die Tr{\"a}gheit eines K{\"o}rpers von seinem Energieinhalt abh{\"a}nging?}, Annalen der Physik 323(13), 639--641, 1905

% A textbook
\bibitem{vallis2006}
G. K. Vallis,
\emph{Atmospheric and Oceanic Fluid Dynamics: Fundamentals and Large-scale Circulation}, Cambridge University Press, 2006

% A website
\bibitem{overleaf}
Overleaf Documentation, https://www.overleaf.com/learn/latex/Main\_Page, accessed 13 August 2020
\end{thebibliography}
}
}

\end{subcolumns}

\end{columns}

\end{document}
